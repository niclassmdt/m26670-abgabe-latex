%Zur Einbingung von Grafiken
\usepackage{graphicx}

%Deutsche Bezeichnungen des Inhaltsverzeichnis, Abbildungsverzeichnis usw.
\usepackage[german]{babel}

%Um Grafiken anzeigen zu lassen, wo sie im tex File geschrieben wurden. \begin{figure}[H]
\usepackage{float}

%Syntax Highlighting
\usepackage{minted}

\usepackage[autostyle,german=quotes]{csquotes}

%Fuer Hintergrundbilder
\usepackage{eso-pic}

%Bilder Transparent machen
\usepackage{transparent}

%Fallunterscheidungen wie etwa gerade und ungerade Seiten
\usepackage{ifthen}

%Einfuegen voll Tabellen
\usepackage{tabularx}

%Literaturverzeichnisse verwenden
\usepackage[style=authortitle, natbib, backend=biber]{biblatex}
\makeatletter% 
\long\def\@makefntext#1{%
  \parindent 1em\noindent \hb@xt@ 1.8em{\hss  \hbox {{\normalfont \@thefnmark }}. }#1}%
\makeatother

%Referenzen erstellen
\usepackage{hyperref}
\hypersetup{
    colorlinks=true,
    linkcolor=black,
    citecolor=black,
    urlcolor=blue,
    pdfborder= 0 0 0
}

%Refernenzieren eines Kapitels mit seinem Namen
\usepackage{nameref}