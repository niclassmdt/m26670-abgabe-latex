\usepackage[T1]{fontenc}
\usepackage[utf8x]{inputenc}
% grafiken einblenden
\usepackage{graphicx}
% deutsche notation Literaturverzeichnis etc. 
\usepackage[english, ngerman]{babel}
% einrückung neuer Paragraphen auf 0 setzen 
\parindent 0pt
\usepackage[T1]{fontenc}
% deutsche Anführungszeichen
\usepackage[autostyle, german=guillemets, german=quotes]{csquotes}
% grafiken gezielt platzieren
\usepackage{float}
% quellcodeimplementation mit highlighting
\usepackage{minted}
% hintergrundbilder
\usepackage{eso-pic}
% transparenz
\usepackage{transparent}
% abwechselnder hintergrund
\usepackage{ifthen}
% tabellen
\usepackage{tabularx}
% literaturverzeichnis
\usepackage{natbib}
% formatierung footmarks
\makeatletter%
\long\def\@makefntext#1{%
    \parindent 1em\noindent \hb@xt@ 1.8em{\hss \hbox{{\normalfont \@thefnmark}}. }#1}%
\makeatother
% linkeinfärbung und referenz
\usepackage{hyperref}
\hypersetup{
    colorlinks = true,
    linkcolor=black,
    citecolor=black,
    urlcolor=blue,
    pdfborder= 0 0 0 % entfernt rahmen, falls noch vorhanden
}
% kapitelreferenzierung
\usepackage{nameref}
% pdfs einfügbar
\usepackage{pdfpages}
% TikZ zum Zeichnen von Bildern
\usepackage{tikz}
%urls in bibtex
\usepackage{url}